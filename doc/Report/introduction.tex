\paragraph{Current situation}
For WantCloud providing transcoding\footnotemark facilities have been
a great source of income. Due to the popularity of the current system,
it is overloaded because it does not scale well during peak usage. In
the existing solution there is only one machine which handles incoming
jobs. The overloaded system therefor has a relatively high number of
outstanding jobs causing it to not meet the deadlines which are
guaranteed by WantCloud's Terms-of-Use. To circumvent the shortcomings
of using only 1 physical machine we will look into using a IaaS as our
platform. 

\paragraph{Related Work}
The pre-existing cloud environment used for the proposed system and
the experiment is the \cite{DAS4} cluster with the
\cite{OpenNebula}-stack on top of \textsc{das4}. OpenNebula provides a
low-level interface for spawning \emph{Virtual Machines} [VM] on which
our workload can then be placed.

For the actual conversion of the media files the \cite{FFmpeg} program
will be used. For the sake of implementation feasability only the
conversion from \cite{\textsc{h264}} to \textsc{dvd}-video is
considered. This software is freely available for everyone to use
under the GPL-license.

As a method for inter-machine communication in the cluster,
\textsc{ssh} is being used. \textsc{ssh} is available in every spawned
machine by default and provides us the means for secure communication
and file-transfer.

\paragraph{Proposed solution}
To be able to cope with demand, a new system setup has been made which
can tap into a pre-existing cloud environment, to scale during peak
usage and thereby load-balance the workload over multiple
machines. For the experiment we've looked at multiple methods for
allocation of the machine resources. By keeping statistics in our
implementation we track the total time it takes for a submitted
media-file to be transcoded and sent back to the submitter. This
metric will hereafter be called the \emph{makespan} of a submitted
job. Another metric is the cost for a job. Leasing a VM in the cloud
costs money -- we investigate the tradeoffs between leasing more VMs
and the effect on the makespan.

To experiment with the proposed solution - a benchmark has been
created which transcodes a particular file on a fully operating VM. To
measure how the system scales, we've used a predefined sample from a
exponentional distribution to simulate arrival times for jobs.

\paragraph{Overview}
In the next section we will elobarate more on the application and
provide more background information. In section \ref{design} the
system's design will be handled so the experiment in section
\ref{experiment} can be understood. After the experiment we will
discuss the findings in section \ref{discussion} and conclude in
section \ref{conclusion}.

\footnotetext{Transcoding: The process of converting a media file
  from one format to another}
