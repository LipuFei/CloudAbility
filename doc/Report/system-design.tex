\subsection{Overview}
From the big picture, our system is designed in a client/server way,
where clients send job requests to the server using a simple
customized language, which specifies things such as which application
to use, which input files to use, etc., and the server parses the
requests, creates a corresponding job and then put it into the pending
job queue. The job is then waiting for dispatch by the scheduler.

The server part of the system consists of the following components:

\paragraph{Listener} 
The listener is responsible for accepting client requests through
socket connection, parsing them, and put the jobs into the pending job
queue.
\paragraph{Scheduler} 
The scheduler is the core thread in the system. It does scheduling
according to the given allocation and provisioning policies and
updates current system status. It will be explained in more details in
the later part.
\paragraph{ResourceManager} 
This component maintains the VM instances on the cloud.
\paragraph{StatisticsManager} 
This module maintains all the statistics data of the system, including
job performances, VM instance performances, etc. It also generates a
final report when the system is shutting down.

Our system is currently only runnable on \textsc{das-4} OpenNebula
platform, but it can be easily modified and extended to support other
cloud platforms because its good flexibility. It is easy to create new
allocation and provisioning policies. All configurations are done
through a configuration file.


\subsection{Resource Management Architecture}
Some features must be explained before we go to more details.

\subsubsection{Jobs}
There are three job queues in the system: \emph{pending job queue},
\emph{running job queue}, and \emph{finished job queue}. It will be
executed when it is assigned to a VM instance. Its execution procedure
is as follows:

\begin{enumerate}
\item Downloading required files (in our case, executable tarball and
  input files).
\item Extracting the executable tarball.
\item Executing the job (converting H.264 video file into
  \textsc{ntsc-dvd}).
\item Uploading the resulting \textsc{dvd} file to the server.
\end{enumerate}

All these operations are done through \textsc{scp} and
\textsc{ssh}. The job thread is also responsible for collecting
performance data, including downloading time, execution time,
uploading time, etc., and once this job is done, it will be put into
the \emph{finished job queue} with its state set to
\statefinished. Later scheduler will remove this job and update the
statistics. If a job fails, its state is set to \statefailed and also put
into the \emph{finished job queue}. The scheduler will handle these
failed jobs according to the given policy.

Although the parameters are configurable for doing other tasks, our
design is not that flexible because this execution sequence is hard
coded in the program. We will consider using script files for job
execution.


\subsubsection{VM instances}
The VM instances are created asynchronously. In OpenNebula, after a VM
instance is allocated, its state becomes \statepending, and you need
to wait until it is \staterunning. At this point, the VM instance is
actually running. However, it is not actually \emph{ready} because
when it becomes \staterunning, it still needs some time to boot the OS and
initialize the system. Only after the system is fully booted can the
VM instance be accessed using \textsc{ssh} to execute jobs.

In our system, we use an asynchronous way to create a VM instance: a
thread called ``VMAgent'' is created every time the system allocates a
VM instance. This thread first allocates the VM instance, and then
does the following things:

\begin{enumerate}
\item VMAgent waits until this VM instance becomes \staterunning.
\item It waits until the VM can be reached by ping.
\item It waits until the VM can be reached by \textsc{ssh}.
\end{enumerate}

After the VM is reachable, it is added into the VM list in the
ResourceManager, and it becomes available for jobs to execute on.

Each VMAgent has a timeout of two minutes. If the operation times out,
the VM instance will be terminated.

\subsubsection{Scheduler}
What the scheduler does is as follows:

\begin{enumerate}
\item It does some regular checks and updates.
\item It picks a job from the pending queue using the specified job
  allocation policy.
\item It picks a VM instance using the specified resource provisioning
  policy.
\item It executes this job on this VM instance.
\end{enumerate}

In the regular check part, the scheduler does miscellaneous tasks,
including:
\begin{itemize}
\item Updating job states, VM instance states, and system states.
\item Checking the \emph{finished job queue}. If a job is finished
  successfully, it is removed and its data is used to update the
  statistics. If it failed, it will be handled according to the given
  allocation policy. For now, the system just puts failed jobs into
  the \emph{pending job queue} again regardless of how many times it
  has failed.
\item Calling the provisioning policy to perform elastic provisioning.
\end{itemize}

In this way, the statistics of the system is continuously recorded,
and flexible allocation policies and elastic provisioning policies can
be implemented.


\subsubsection{Reliability}
Our system keeps tracking on every job and VM instance. During the
shutdown sequence, it first stops all executing jobs. Then, it
terminates all allocated VM instances, so that no VM instance would be
running afterwards. After that, it checks jobs in all the queues, and
put them into the statistics module, which finally creates a report
and outputs to a file.

Besides that, our system also keeps separate log files for the system
itself and each jobs respectively. So it is easy to debug the system.


\subsection{System Policies}
We have implemented one job allocation policy and two resource
provisioning policies.

\subsubsection{Allocation Policy}
The job allocation policy we implemented is a First-Come-First-Serve
(\textsc{fcfs}) policy. It always picks the job in the pending queue
with the earliest arrival time.

\subsubsection{Provisioning Policies}
We implemented two provisioning policies: \policystatic policy and
\policysimpleelastic policy.

\policystatic policy allocated a specified number of VM instances when the
system starts, and this number will not vary over time. This policy
can not adapt itself to the changing environment.

\policysimpleelastic policy is on-demand-like elastic policy, which has
three parameters:

\begin{itemize}
\item \emph{minvms}: The minimum number of VM instances the system
  must have.
\item \emph{maxmvs}: The maximum number of VM instances the system can
  have.
\item \emph{threshold}: A threshold value, which will be explained
  later.
\end{itemize}

Each time the system statuses have been updated by the scheduler, this
provisioner is called, and it checks the number of pending jobs. If
this number is larger than the \emph{threshold}, it allocates one more
VM instance until either the total number of VM instances in the
system equals the sum of pending jobs and running jobs or it reaches
the \emph{maxvms}. However, when the pending job queue becomes empty,
this policy will find one idle VM instance and release it.

It elastic policy is a simple on-demand policy is because it doesn't
take into account how long a VM instance has been idle. It simply
terminates one if it is currently idle. A main drawback of this policy
can be illustrated in this scenario: Suppose there are 7 jobs running,
no pending jobs, and 8 VMs in the system. So this policy will remove
one idle VM. If a job arrives right after the policy removes a VM,
this job has to wait for a long time to obtain an available VM.


